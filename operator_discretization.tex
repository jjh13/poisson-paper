\section{Operator Discretization}
\label{sec:opDisc}

The asymptotic optimality criterion amounts to requiring that the
residue error kernel asymptotically behaves like the minimum error
kernel, i.e. $E_{\mathsf{mod}}(\vect{\omega}) \approx
E_{\mathsf{min}}(\vect{\omega}) = O(\norm{\vect{\omega}}^{2k})$ around
$\vect{\omega} = 0$. In this section, we consider two discretization
models: an \emph{interpolative} model that is based on an
interpolation prefilter obtained from the samples of $\varphi$, and a
more accurate two-stage \emph{quasi-interpolative} model that makes
use of a higher order auxiliary generator $\psi$. Both scenarios yield
digital filters that can be readily applied to the samples in the
Fourier domain via the MDST.

\subsection{Interpolative Model}
\label{sec:interpModel}
Suppose that our desired approximation space is spanned by generator
$\varphi$ where $\order{\lattice{L}}{\varphi} =
k$. From~\eqref{eq:fourierSolution}, the operator that we need to
discretize is $\invlaplacian$. In order to obtain a discretization, we
make use of the filter $\ft{P}_{\varphi}(\vect{\omega})$ that is
obtained from the samples of $\varphi$ that lie on the nodes of
$\lattice{L}$. Recall that the inverse filter
$(\ft{P}_{\varphi}(\vect{\omega}))^{-1}$ is needed to make the generator $\varphi$
interpolating (\SC{sec:interpQuasiInterp}).

Let us denote the discretization of $\invlaplacian$ on the normalized lattice
$\lattice{L}$ as $l^{-1}$ so that the approximate solution on the lattice
$\lattice{L}_h$ can be written as
\begin{equation}
\appf{V}(\vect{x}) = h^2 \sum_{j=1}^{2^s N}
\underbrace{(f \pconv l^{-1}_h)}_{c}[\vect{x}_j]
\varphi_p(\frac{\vect{x}-\vect{x}_j}{h}).
\label{eq:approxHomogeneous}
\end{equation}
Using the
residue error kernel~\eqref{eq:opResidue}, the optimality criterion
boils down to requiring that
\begin{equation}
  \ft{A}_{\varphi}(\vect{\omega})
  \bigl|
  4 \pi^2 \norm{\vect{\omega}}^2 \wft{L^{-1}}(\vect{\omega})
  + 
  \ft{\dual{\varphi}}(\vect{\omega}) 
  \bigr|^2
  = 
  O(\norm{\vect{\omega}}^{2k}).
  \label{eq:opCriterionHomo}
  \end{equation}
  In order to arrive at this result, we have extended the operator
  $\invlaplacian$ to $L_2(\reel^s)$ where $\ftpairs{\invlaplacian}{(-4
    \pi^2 \norm{\vect{\omega}}^2)^{-1}}$. The reason for choosing
  $l^{-1}$ as a discretization of $\invlaplacian$ will become apparent
  in light of the following proposition which gives us a way of
  reformulating the problem of designing $l^{-1}$, a filter that
  discretizes $\invlaplacian$, into a simpler problem of designing the
  filter $l := (l^{-1})^{-1}$ that discretizes the Laplacian
  $\laplacian$.

\begin{prop}
\label{prop:1}
  Let the digital filter $\ft{L}(\vect{\omega})$ be given by the
  combined filter $\ft{L}(\vect{\omega}) =
  \ft{P}_{\varphi}(\vect{\omega})\ft{\Lambda}(\vect{\omega})$, where
  $\ftpairs{\ft{\Lambda}(\cdot)}{\lambda[\cdot]}$ is a digital filter that satisfies
  \begin{equation}
    \ft{\Lambda}(\vect{\omega}) = -4\pi^2\norm{\vect{\omega}}^2 
    + O(\norm{\vect{\omega}}^{k+2}).
    \label{eq:discreteLaplacian}
  \end{equation}
  Then the combined inverse filter $\wft{L^{-1}}(\vect{\omega}) =
  (\ft{P}_{\varphi}(\vect{\omega})\ft{\Lambda}(\vect{\omega}))^{-1}$
  provides a $k$-th order asymptotically optimal discretization of the
  inverse Laplacian $\invlaplacian$.
\end{prop}
The proof is provided in Appendix~\ref{app:proofs}.

It is not hard to see that this method is essentially a collocation method with
respect to the trial space spanned by the generator $\varphi_p$. If we apply the
discrete Laplacian filter $\ftpairs{\lambda}{\ft{\Lambda}}$ to the
coefficients in approximation~\eqref{eq:approxHomogeneous}, we obtain
\begin{equation}
  (\laplacian \appf{V})(\vect{x}) \approx
  \sum_{j=1}^{2^s N}
  (f \pconv p_{\varphi}^{-1})[\vect{x}_j]
  \varphi_p(\frac{\vect{x}-\vect{x}_j}{h}),
\end{equation}
where
$\ftpairs{p_{\varphi}^{-1}[\cdot]}{(\ft{P}_{\varphi}(\cdot))^{-1}}$. Owing to 
the fact that $p^{-1}_{\varphi}$ is an interpolation filter, this approximation,
when evaluated at the lattice sites contained in $\oucube^s$, will yield the
samples of $f$.

\subsection{Quasi-interpolative Model}
% The model described in the previous section heavily relies on the
% interpolation prefilter $1/\ft{P}_{\varphi}$ that ensures that the
% discretized Laplacian of the approximate solution, exactly reproduces
% the sample values of $f$ at the lattice sites. For this reason, it can
% also be regarded as a collocation method with respect to the trial
% space spanned by the generator $\varphi_P$. 

We now describe a two-stage model that relies on two spaces: a desired
target space $\aspace{\lattice{L}_h}{\varphi_p}$ spanned by the
generator $\varphi_p$ where $\order{\lattice{L}}{\varphi} = k$, and an
auxiliary space $\aspace{\lattice{L}_h}{\psi_p}$ spanned by the
generator $\psi_p$ where $\order{\lattice{L}}{\psi} \ge k$. In the
first stage, we use the interpolative model from the previous section
to seek an auxiliary solution that lies in
$\aspace{\lattice{L}_h}{\psi_p}$. Then, the auxiliary solution is
orthogonally projected to the target space
$\aspace{\lattice{L}_h}{\varphi_p}$. This method is similar to our
two-stage gradient estimation framework~\cite{hossain10}. Owing to the
shift-invariant nature of the spaces, the orthogonal projection is
tantamount to applying a digital filter to the first-stage
approximation coefficients as described below.

Let us denote the first-stage approximation coefficients as
$c_{\psi}[\vect{x}_j]$. These coefficients are obtained by applying the discrete
operator $l_{\psi}^{-1}$ to the samples of $f$ as given
in~\eqref{eq:approxHomogeneous}. The discretization $l_{\psi}^{-1}$ is obtained
from the interpolative model where the interpolation prefilter $p_\psi$ is derived
from the samples of the first-stage generator $\psi$. Now, if $\psi$ is chosen
to have a sufficiently high order, then we know that this auxiliary
approximation will be very close to the true solution. Therefore, we can use it
to find an approximation in our desired target space
$\aspace{\lattice{L}_h}{\varphi}$. In particular, we orthogonally project the
auxiliary approximation onto the target space so as to minimize the $L_2$ error.
The orthogonal projection is obtained by taking inner products of the auxiliary
approximation with the lattice translates of $\dual{\varphi}_p$: the dual of the
target space generator $\varphi_p$. In particular, the second-stage
approximation can be written as
\begin{equation}
  \begin{split}
  & \appf{V}(\vect{x}) = \\
  & \sum_{j =1}^{2^s N} 
  \Bigl(
    \sum_{k = 1}^{2^s N}
    c_\psi \innerproduct{\psi_p(\frac{\cdot - \vect{x}_k}{h})}
    {h^{-s}\dual{\varphi}_p(\frac{\cdot - \vect{x}_j}{h})}
  \Bigr)
  \varphi_p(\frac{\vect{x}-\vect{x}_j}{h}).
\end{split}
\end{equation}
Assuming that $\psi$ and $\varphi$ are symmetric admissible
generators, this is equivalent to the digital filtering scheme
\begin{equation}
  \appf{V}(\vect{x}) = \sum_{j=1}^{2^s N} (c_\psi \pconv q_h)[\vect{x}_j]
  \varphi_p(\frac{\vect{x}-\vect{x}_j}{h}),
  \label{eq:quasiApprox}
\end{equation}
where the weights of the filter $q_h$ are obtained according to
\begin{equation}
  q_h[h\mat{L}\vect{k}] =  q[\mat{L}\vect{k}] := (\psi \ast
  \dual{\varphi})(\mat{L}\vect{k}).
  \label{eq:quasiWeights}
\end{equation}
Here, $(\psi \ast \dual{\varphi})$ denotes the continuous convolution
of the two generators $\psi$ and $\dual{\varphi}$ in $L_2(\reel^s)$.

An approximation model is said to be
quasi-interpolating of order $k$ if it is capable of reproducing all
multivariate polynomials in $\Pi_{k-1}(\reel^s)$, or  equivalently,
$E_{\mathsf{res}}(\vect{\omega}) = O(\norm{\vect{\omega}}^{2m})$  ($m \ge k$)
where $E_{\mathsf{res}}$ is as defined in~\eqref{eq:scErrorKernel}. We retain
the same terminology and say that an approximate solution to the Poisson
equation~\eqref{eq:PoissonEquation} is quasi-interpolating of order $k$ if it
satisfies $E_{\mathsf{mod}}(\vect{\omega}) = O(\norm{\vect{\omega}}^{2m})$,
where $m \ge k$. The following proposition gives us a way of choosing the
auxiliary space $\aspace{\lattice{L}_h}{\psi_p}$ so as to ensure that the
resulting approximation~\eqref{eq:quasiApprox} is quasi-interpolating of order
$k$.

\begin{prop}
  Let $\aspace{\lattice{L}_h}{\psi_p}$ and
  $\aspace{\lattice{L}_h}{\varphi_p}$ be the auxiliary and target
  approximation spaces respectively, with $\order{\lattice{L}}{\psi} = m$ and
  $\order{\lattice{L}}{\varphi} = k$. The approximation $\appf{V}$ obtained
  through \eqref{eq:quasiApprox} is quasi-interpolating of order $k$
  if $m \ge k$.
\label{prop:2}
\end{prop}
Please refer to Appendix~\ref{app:proofs} for the proof.\\
We note that when the auxiliary space is chosen to be the same as the
target space, i.e. when $\psi = \varphi$, then $q[\mat{L}\vect{k}] =
\delta[\vect{k}]$, this scheme reduces to the interpolative model
described in the previous section.
