\section{Related Work}
.
Surface reconstruction is an active area of research in computer graphics. It is beyond the scope of this work to touch on all the relevant research surrounding the subject. Thus, we only very briefly review some relevant contributions. Those seeking a more in-depth review may look to a recent survey of the subject \cite{sve05ss}, we focus mainly on some of the more related recent developments. 

Surface reconstruction algorithms generally fall under two categories:
\emph{implicit} algorithms and \emph{explicit} algorithms. The common
theme of implicit algorithms is the desire to find an implicit
function $f : \realn^3 \mapsto \realn $, for which the iso-surface
defined by $f(\mathbf{x})=s$ (for some iso-value $s \in \realn$)
represents an approximation to the original surface. Explicit
algorithms rely on combinatorial techniques, inferring meshes through
a set predefined logic or rules.

In terms of explicit algorithms, there exist many Delaunay style
algorithms in which a triangulation is constructed using only a subset
of the original point-set \cite{delaunay}. The Power Crust
\cite{powercrust} and Cocone \cite{cocone} algorithms are two
well-known techniques that fall under this category. These techniques
often aim to interpolate the input point set, thus in the presence of
noise, tend to over-fit the surface. When the input data are noisy,
implicit approaches often perform better, as they inherently attempt
to fit locally smooth basis functions to the data, averaging out any
noise. Reconstructing either a distance function or an indicator
function is common to these approaches.

One of the most well known implicit algorithms is the %Graph Cut \tm{this is not graph cuts!!}
algorithm of by Hoppe et al.~\cite{hoppecut}. In this work a set of
tangent planes is constructed to create an approximation to a
surface's distance function. A more recent work, is the Smoothed
Signed Distance Field reconstruction technique \cite{ssdrecon}, in
which an oriented point set imposes constraints on the reconstructed
function, its gradient, and its Hessian. These constraints are
tantamount to requiring that the reconstructed function be an
approximation to the distance field of the original model. Other
techniques attempt to reconstruct a signed distance function within a
function space spanned by radial basis functions \cite{radial}.
Recently, there has also been research that incorporates point normals
into the radial basis reconstruction framework \cite{hermite}.

Another approach is to find an approximate indicator function of the
original model. A well known technique \cite{fftk} transforms the
smoothed gradient to the Fourier domain, inverts it, then reconstructs
it in the spatial domain. This approach often deals well with
incomplete, missing or otherwise noisy data, with the guarantee of
creating a water tight surface. However, it involves reconstructing
the indicator on a regular grid, which becomes very memory intensive
as the grid is refined.

Poisson surface reconstruction techniques make the observation that
the Fourier method is equivalently stated as a spatial Poisson problem
\cite{Kazhdan06,screenedk}. Moreover, they seek to rectify the memory
limitations of said work, while still maintaining the benefits it
provides. However, current techniques based on this idea have been
shown to over-smooth the initial point-set, and thus the resulting
surface. \cite{reconbench}.

It is also possible to reconstruct the indicator of a point set within
a wavelet basis \cite{wavelet}. In this work, an approximate indicator
function of the model is projected onto a compact wavelet basis. This
leads to a simple and efficient algorithm even when the data sets
are massive. However the method's speed comes at a price, as the
resulting surfaces are often non-smooth, and display many ``jagged''
artifacts (although a form of smoothing is applied to the resulting
indicator function in an attempt to rectify this.)

Our approach aims to solve the Poisson surface reconstruction problem
in the setting of shift-invariant spaces. A crucial step in this
regard is the accurate estimation of the divergence of the gradient
field from the oriented point-set. Radial basis functions (RBFs) are
generally used in the context of the reconstruction of scattered data. 
However, RBF techniques are often computationally
expensive, requiring the solution of large unstable systems of
equations (for exact interpolation.) A proposed solution to this
problem is to ``\emph{re-sample}'' the input data onto a regular
shift-invariant space . This allows one to use efficient compact
reconstruction kernels and data processing techniques
\cite{variational,onvari}. Recently, an extension of this idea to box
spline spaces \cite{xu2012rec} proposes a variational reconstruction
framework in which the BCC lattice consistently outperforms the
Cartesian lattice. Our method takes a similar variational approach and
seeks to construct a smoothed lattice-based approximation of the
gradient of the indicator function (\SC{sec:obtainingV}). Our employed
constraints force the gradient to be close to zero away from the
surface, while maintaining a certain degree of smoothness in the
resulting approximation. This is preferable in comparison to other
works \cite{fftk,Kazhdan06,screenedk} in which each sample is
trilinearly distributed about either a grid or an oct-tree
respectively, and is known to over-smooth data.

Having obtained a high-quality lattice-based approximation of the
gradient field, we invoke the theory behind shift-invariant spaces,
and accurately estimate the divergence by applying derivative filters
that are designed to harness the full approximation capabilities of
the target space (\SC{sec:divergence}). In a similar manner, we then
solve the Poisson equation by applying an \emph{inverse} Laplacian
filter that is tailored to match the approximation order of the target
space (\SC{sec:recoveringX}).
